\documentclass[11pt]{article}

\usepackage{amsmath, amsthm, amssymb, mathtools, commath}
\usepackage{datetime, geometry, xcolor, tcolorbox, graphicx, calc, cancel, tabularx
	%enumitem
}
\usepackage[tikz]{mdframed}
\usepackage{tikz}

\geometry{
	right=0.5in,
	left=0.5in,
	top=1in,
	bottom=0.8in
}

\usepackage{fancyhdr}

\pagestyle{fancy}

\fancyhf{}
\lhead{Problem Set 4. Discrete Random Variables}
\rhead{Vinesh Benny}
\rfoot{\thepage}

%\renewcommand{\headrulewidth}{0pt}

\title{STAB52 -- Problem Set 2. Counting}
\author{Vinesh Benny}
\renewcommand\theenumii{\roman{enumii}}
\renewcommand\labelenumii{\theenumii}      %This changes the labelling for second level of enum globally. Should use enumitem instead.

\renewcommand{\qedsymbol}{$ \blacksquare $}

\definecolor{correct}{RGB}{37,150,190}

\mdfsetup{%
	middlelinecolor=blue!40,
	middlelinewidth=1pt,
	backgroundcolor=blue!25!purple!10!,
	roundcorner=5pt,
	%	innerleftmargin=1cm,
	%	innerrightmargin=1cm
}

%\usepackage{fourier}
\renewcommand{\rmdefault}{bch}
%\usepackage[sfdefault]{roboto}  %% Option 'sfdefault' only if the base font of the document is to be sans serif
%\usepackage[T1]{fontenc}
%\renewcommand{\familydefault}{\sfdefault}

%\pagecolor[rgb]{0,0,0} %black    % to change page to night mode
%\color[rgb]{0.5,0.5,0.5} %grey

\begin{document}
\begin{enumerate}
	\item (Evans 2.3.1) Suppose that we roll two fair six-sided dice, and let $ Y $ be the sum of the two numbers showing. Compute $ P(Y = y) $ for every real number $ y $.
		\begin{mdframed}
			The sample space $ S $  would be $ S = $ \textbraceleft (1, 1), (1, 2), (1, 3), (1, 4), (1, 5), (1, 6),\\
			(2, 1), (2, 2), (2, 3), (2, 4), (2, 5), (2, 6),\\
			(3, 1), (3, 2), (3, 3), (3, 4), (3, 5), (3, 6),  \\
			(4, 1), (4, 2), (4, 3), (4, 4), (4, 5), (4, 6),  \\
			(5, 1), (5, 2), (5, 3), (5, 4), (5, 5), (5, 6),  \\
			(6, 1), (6, 2), (6, 3), (6, 4), (6, 5), (6, 6) \textbraceright with 36 elements. Therefore, $ Y \in \left\lbrace 1, 2, \ldots, 12 \right\rbrace $.\\
			$ P(Y = y) $ for every real number $ y $:\\[-20pt]
			\begin{center}
				\begin{tabular}{*{12}{| c }|}
					\hline
					y            & 2                & 3                & 4                & 5                & 6                & 7                & 8                & 9                & 10               & 11               & 12               \\ \hline
					$ P(Y = y) $ & $ \frac{1}{36} $ & $ \frac{2}{36} $ & $ \frac{3}{36} $ & $ \frac{4}{36} $ & $ \frac{5}{36} $ & $ \frac{6}{36} $ & $ \frac{5}{36} $ & $ \frac{4}{36} $ & $ \frac{3}{36} $ & $ \frac{2}{26} $ & $ \frac{1}{36} $ \\ \hline
				\end{tabular}
			\end{center}
		\end{mdframed}

	\item Suppose that a bowl contains 100 chips: 30 are labelled 1, 20 are labelled 2, and 50 are labelled 3. The chips are thoroughly mixed, a chip is drawn, and let X be the number on the chip.
		\begin{enumerate}
			\item Compute $ P(X = x) $ for every real number $ x $.
				\begin{mdframed}
					$ X \in \left\lbrace 1, 2, 3 \right\rbrace  $ since those are the numbers for the chips.\\
					$ P(X = x) $ for every real number $ x $:\\[-20pt]
					\begin{center}
						\begin{tabular}{*{4}{| c }|}
							\hline
							x            & 1                & 2                 & 3                \\ \hline
							$ P(X = x) $ & $ \frac{3}{10} $ & $ \frac{2}{10}  $ & $ \frac{5}{10} $ \\ \hline
						\end{tabular}
					\end{center}
				\end{mdframed}
			\item Suppose the first chip is replaced, a second chip is drawn, and let $ Y $ be the number on the second chip. Compute $ P(Y = y) $ for every real number $ y $.
				\begin{mdframed}
					Since the first chip is replaced, $ Y $ has the same probabilities as $ X $. i.e. $ P(Y = y) $ for every real number $ y $:\\[-20pt]
					\begin{center}
						\begin{tabular}{*{4}{| c }|}
							\hline
							y            & 1                & 2                 & 3                \\ \hline
							$ P(Y = y) $ & $ \frac{3}{10} $ & $ \frac{2}{10}  $ & $ \frac{5}{10} $ \\ \hline
						\end{tabular}
					\end{center}
				\end{mdframed}
			\item Let $ W =X+Y $ and compute $ P(W = w) $ for every real number $ w $.
				\begin{mdframed}
					$ W \in \left\lbrace 2, 3, 4, 5, 6 \right\rbrace $\\
					$ P(W = w) = \left\{ \begin{array}{lll}
						2 &= (1, 1)	&= (0.3 \cdot 0.3) = 0.09\\
						3 &= (1,2) \text{ or } (2,1)	&= (0.2 \cdot 0.3) + (0.3 \cdot 0.2) = 0.12\\
						4 &= (1,3) \text{ or } (3,1) \text{ or } (2,2) &= (0.3 \cdot 0.5) + (0.5 \cdot 0.3) + (0.2 \cdot 0.2) = 0.34\\
						5 &= (2,3) \text{ or } (3,2) &= (0.2 \cdot 0.5) + (0.5 \cdot 0.2) = 0.2\\
						6 &= (3,3) &= (0.5 \cdot 0.5) = 0.25
					\end{array}\right.  $\\
				i.e. $ P(W = w) = \left\{ \begin{array}{lr}
					0.09 &, w = 2\\
					0.12 &, w = 3\\
					0.34 &, w = 4\\
					0.2 &, w = 5\\
					0.25 &, w = 6
				 \end{array} \right.$
				\end{mdframed}
		\end{enumerate}

	\newpage

	\item Suppose that a bowl contains 10 chips, each uniquely numbered 0 through 9. The chips are thoroughly mixed, one is drawn and let $ X_1 $ be the number on it. This chip is then replaced in the bowl, and a second chip is drawn, with number $ X_2 $. Compute $ P(W = w) $ for every real number $ w $ when $ W = X_1 + 10X_2 $.
		\begin{mdframed}
			Any number from 0--99 can be made since chips are replaced, \\
			i.e. $ X_1 \in \left\{0, 1, \ldots ,9\right\}, X_2\in \left\{0, 1, \ldots ,9\right\},  W \in \left\{ 00, 01, \ldots ,99\right\}  $. \\
			Therefore the probability of any $ w $ is the probability of getting the corresponding $ x_1 $ \emph{and} $ x_2 $ value.\\
			$ P(W = w) $, where $ w = x_2x_1 $, $ = P(X_2 = x_2) \cap P(X_1 = x_2) = \frac{1}{10} \cdot \frac{1}{10} = \fbox{$ \frac{1}{100} $}$
		\end{mdframed}

	\item (Blitzstein: \S 3, Q1) People are arriving at a party one at a time. While waiting for more people to arrive they entertain themselves by comparing their birthdays. Let $ X $ be the number of people needed to obtain a birthday match, i.e. before the $ X $th person arrives there are no two people with the same birthday, but when person $ X $ arrives there is a match. Find the distribution of X.
\end{enumerate}
\end{document}