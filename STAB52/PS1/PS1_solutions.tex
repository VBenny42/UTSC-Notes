\documentclass[11pt]{article}

\usepackage{amsmath, amsthm, amssymb, mathtools, commath}
\usepackage{datetime, geometry, xcolor, tcolorbox
	%enumitem
}
\usepackage[tikz]{mdframed}

\geometry{
	right=0.5in,
	left=0.5in,
	top=1in,
	bottom=0.8in
}

\usepackage{fancyhdr}

\pagestyle{fancy}

\fancyhf{}
\lhead{Problem Set 1. Probability Basics}
\rhead{Vinesh Benny}
\rfoot{\thepage}

%\renewcommand{\headrulewidth}{0pt}

\title{STAB52 -- Problem Set 1. Probability Basics}
\author{Vinesh Benny}

\renewcommand\theenumii{\roman{enumii}}
\renewcommand\labelenumii{\theenumii}      %This changes the labelling for second level of enum globally. Should use enumitem instead.

\definecolor{correct}{RGB}{37,150,190}

\mdfsetup{%
	middlelinecolor=blue!40,
	middlelinewidth=1pt,
	backgroundcolor=blue!25!purple!10!,
	roundcorner=5pt,
%	innerleftmargin=1cm,
%	innerrightmargin=1cm
}

%\usepackage{fourier}
\renewcommand{\rmdefault}{bch}
%\renewcommand{\familydefault}{\sfdefault}

%\pagecolor[rgb]{0,0,0} %black    % to change page to night mode
%\color[rgb]{0.5,0.5,0.5} %grey

\begin{document}
%	\maketitle
	\begin{enumerate}
		\item (Setting up sample spaces) Assume two standard six-sided dice are rolled one after the other.
			\begin{enumerate} % [label={\roman*}]
				\item Restrict attention to the first roll, and list all possible outcomes of this random experiment.
					\begin{mdframed}
						First roll can either be 1, 2, 3, 4, 5, or 6.  $\left( S =  \left\lbrace x \in \mathbb{N} \mid 1 \leq x \leq 6  \right\rbrace \right) $\\
						We do not care about the outcomes of the second roll.
					\end{mdframed}
				\item Now consider both rolls, and list the outcomes of the most general sample space you can define.
					\begin{mdframed}
						Each roll can either be 1, 2, 3, 4, 5, or 6. Therefore outcomes are: $ \left( S =  \left\lbrace x, y \in \mathbb{N} \mid 1 \leq x, y \leq 6  \right\rbrace \right) $ or\\
						\textbraceleft (1, 1), (1, 2), (1, 3), (1, 4), (1, 5), (1, 6),\\
						(2, 1), (2, 2), (2, 3), (2, 4), (2, 5), (2, 6),\\
						(3, 1), (3, 2), (3, 3), (3, 4), (3, 5), (3, 6),  \\
						(4, 1), (4, 2), (4, 3), (4, 4), (4, 5), (4, 6),  \\
						(5, 1), (5, 2), (5, 3), (5, 4), (5, 5), (5, 6),  \\
						(6, 1), (6, 2), (6, 3), (6, 4), (6, 5), (6, 6) \textbraceright
					\end{mdframed}
			\end{enumerate}

		\item (Set theory) Consider two arbitrary events $ A, B \in S $ and describe the following events using set operations.
			\begin{enumerate}
				\item Both events occur.
					\begin{mdframed}
						$ A \cap B $
					\end{mdframed}
				\item At least one event occurs.
					\begin{mdframed}
						$ A \cup B $
					\end{mdframed}
				\item Neither event occurs.
					\begin{mdframed}
						$ \neg \left( A \cup B \right)  $
					\end{mdframed}
				\item Only event $ A $ but not event $ B $ occurs, i.e. $ \left\lbrace s \in S : s \in A \text{ and } s \notin B \right\rbrace $.\\
				Note: this set operation is called \textit{difference} and is denoted A -- B (minus) or A\textbackslash B (back-slash).
					\begin{mdframed}
						$ A \cap \neg B $
					\end{mdframed}
				\item Exactly one event occurs.\\
				Note: this set operation is called \textit{symmetric difference} and is denoted by $ A\triangle B $. In logic it is also called \textit{exclusive OR} (XOR).
				\begin{mdframed}
					$ \left( A \cap \neg B \right) \cup \left( \neg A \cap B \right) \cap \neg \left( A \cup B \right) $
				\end{mdframed}
			\end{enumerate}
	\end{enumerate}
\end{document}