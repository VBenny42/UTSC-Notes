\documentclass[11pt]{article}

\usepackage{amsmath, amsthm, amssymb, mathtools, commath}
\usepackage{datetime, geometry, xcolor, tcolorbox, graphicx
	%enumitem
}
\usepackage[tikz]{mdframed}
\usepackage{tikz}

\geometry{
	right=0.5in,
	left=0.5in,
	top=1in,
	bottom=0.8in
}

\usepackage{fancyhdr}

\pagestyle{fancy}

\fancyhf{}
\lhead{Problem Set 1. Probability Basics}
\rhead{Vinesh Benny}
\rfoot{\thepage}

%\renewcommand{\headrulewidth}{0pt}

\title{STAB52 -- Problem Set 1. Probability Basics}
\author{Vinesh Benny}

\renewcommand\theenumii{\roman{enumii}}
\renewcommand\labelenumii{\theenumii}      %This changes the labelling for second level of enum globally. Should use enumitem instead.

\renewcommand{\qedsymbol}{$ \blacksquare $}

\definecolor{correct}{RGB}{37,150,190}

\mdfsetup{%
	middlelinecolor=blue!40,
	middlelinewidth=1pt,
	backgroundcolor=blue!25!purple!10!,
	roundcorner=5pt,
%	innerleftmargin=1cm,
%	innerrightmargin=1cm
}

%\usepackage{fourier}
\renewcommand{\rmdefault}{bch}
%\renewcommand{\familydefault}{\sfdefault}

%\pagecolor[rgb]{0,0,0} %black    % to change page to night mode
%\color[rgb]{0.5,0.5,0.5} %grey

\begin{document}
%	\maketitle
	\begin{enumerate}
		\item (Setting up sample spaces) Assume two standard six-sided dice are rolled one after the other.
			\begin{enumerate} % [label={\roman*}]
				\item Restrict attention to the first roll, and list all possible outcomes of this random experiment.
					\begin{mdframed}
						First roll can either be 1, 2, 3, 4, 5, or 6.  $\left( S =  \left\lbrace x \in \mathbb{N} \mid 1 \leq x \leq 6  \right\rbrace \right) $\\
						We do not care about the outcomes of the second roll.
					\end{mdframed}
				\item Now consider both rolls, and list the outcomes of the most general sample space you can define.
					\begin{mdframed}
						Each roll can either be 1, 2, 3, 4, 5, or 6. Therefore outcomes are: $ \left( S =  \left\lbrace x, y \in \mathbb{N} \mid 1 \leq x, y \leq 6  \right\rbrace \right) $ or\\
						\textbraceleft (1, 1), (1, 2), (1, 3), (1, 4), (1, 5), (1, 6),\\
						(2, 1), (2, 2), (2, 3), (2, 4), (2, 5), (2, 6),\\
						(3, 1), (3, 2), (3, 3), (3, 4), (3, 5), (3, 6),  \\
						(4, 1), (4, 2), (4, 3), (4, 4), (4, 5), (4, 6),  \\
						(5, 1), (5, 2), (5, 3), (5, 4), (5, 5), (5, 6),  \\
						(6, 1), (6, 2), (6, 3), (6, 4), (6, 5), (6, 6) \textbraceright
					\end{mdframed}
			\end{enumerate}

		\item (Set theory) Consider two arbitrary events $ A, B \in S $ and describe the following events using set operations.
			\begin{enumerate}
				\item Both events occur.
					\begin{mdframed}
						$ A \cap B $
					\end{mdframed}
				\item At least one event occurs.
					\begin{mdframed}
						$ A \cup B $
					\end{mdframed}
				\item Neither event occurs.
					\begin{mdframed}
						$ \neg \left( A \cup B \right)  $
					\end{mdframed}
				\item Only event $ A $ but not event $ B $ occurs, i.e. $ \left\lbrace s \in S : s \in A \text{ and } s \notin B \right\rbrace $.\\
				Note: this set operation is called \textit{difference} and is denoted A -- B (minus) or A\textbackslash B (back-slash).
					\begin{mdframed}
						$ A \cap \neg B $
					\end{mdframed}
				\item Exactly one event occurs.\\
				Note: this set operation is called \textit{symmetric difference} and is denoted by $ A\triangle B $. In logic it is also called \textit{exclusive OR} (XOR).
				\begin{mdframed}
					$ \left( A \cap \neg B \right) \cup \left( \neg A \cap B \right) $
				\end{mdframed}
			\end{enumerate}

		\newpage
		\item (Blitzstein: \S 1, Q42-43) For arbitrary events A, B use the probability axioms to show:\\
			{\footnotesize The definition of the difference ($ - $) and symmetric difference ($ \triangle $) set operations are given in problem 2.iv and 2.v.}
			\begin{enumerate}
				\item $ P(A - B)=P(A) - P(A \cap B) $.
					\begin{mdframed}
						\begin{proof}
							Recall: $ A = (A \cap B) \cup (A \cap B^c) = (A \cap B) \cup (A - B)$, where $ (A \cap B) $ and $ (A - B) $ are disjoint.\\
							$ \therefore P(A) = \underbrace{P(A \cap B) + P(A - B)}_{\text{By the additivity axiom.}} \implies  P(A - B)=P(A) - P(A \cap B ) \qedhere$
						\end{proof}
					\end{mdframed}
				\item $P(A \triangle B)=P(A)+P(B) - 2P(A \cap B) $.
					\begin{mdframed}
						\begin{proof}
							Recall: $ A - B $ and $ B - A $ are disjoint.
							\begin{align*}
								P(A \triangle B) &= P((A - B) \cup (B - A))\\
								&= P(A-B) + P(B-A) &\text{By the additivity axiom.}\\
								&= \left( P(A) - P(A \cap B) \right) + \left( P(B) - P(A \cap B) \right) &\text{By part i}\\
								\therefore &=P(A)+P(B) - 2P(A \cap B) &\qedhere
							\end{align*}
						\end{proof}
					\end{mdframed}
			\end{enumerate}

		\item Consider three arbitrary sets $ A, B, C $ represented by colored circles in the Venn diagram below.
			\begin{mdframed}[default, userdefinedwidth=4.8cm]
				\resizebox{4cm}{!}{\begin{tikzpicture}
					\definecolor{vennred}{HTML}{ff989b}
					\definecolor{vennblue}{HTML}{90ccff}
					\definecolor{venngreen}{HTML}{cbe196}
					\begin{scope}[blend group = soft light]
%						\fill[red!30!white!90]   ( 150:1.2) circle (2);
						\fill[vennred]   ( 150:1.2) circle (2);
%						\fill[blue!30!white!90] (270:1.2) circle (2);
						\fill[vennblue] (270:1.2) circle (2);
%						\fill[green!30!white!90]  (30:1.2) circle (2);
						\fill[venngreen]  (30:1.2) circle (2);
					\end{scope}
					\node at (135:3.4) {$ A $};
					\node at (45:3.4) {$ B $};
					\node at (240:3.3)	{$ C $};
					\node at ( 150:2)    {$ F $};
					\node at (90:1.5) 	{$ G $};
					\node at (210:1.5) 	{$ I $};
					\node [] {$ H $};
				\end{tikzpicture}}
			\end{mdframed}
			\begin{enumerate}
				\item Describe the strictly pink area $ F $ using set operations on $ A, B, C $.
					\begin{mdframed}
						$ F = A \cap \neg (B \cup C) $
					\end{mdframed}
				\item Describe the strictly brown area $ G $ using set operations on $ A, B, C $.
					\begin{mdframed}
						$ G = A \cap B \cap \neg C $
					\end{mdframed}
			\end{enumerate}
	\end{enumerate}
\end{document}