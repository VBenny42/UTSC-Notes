\documentclass[11pt]{article}

\usepackage{amsmath, amsthm, amssymb, mathtools, commath}
\usepackage{datetime, geometry, xcolor, tcolorbox, graphicx, calc, cancel
	%enumitem
}
\usepackage[tikz]{mdframed}
\usepackage{tikz}

\geometry{
	right=0.5in,
	left=0.5in,
	top=1in,
	bottom=0.8in
}

\usepackage{fancyhdr}

\pagestyle{fancy}

\fancyhf{}
\lhead{Problem Set 3. Conditional Probability \& Bayes Rule}
\rhead{Vinesh Benny}
\rfoot{\thepage}

%\renewcommand{\headrulewidth}{0pt}

\title{STAB52 -- Problem Set 2. Counting}
\author{Vinesh Benny}
\renewcommand\theenumii{\roman{enumii}}
\renewcommand\labelenumii{\theenumii}      %This changes the labelling for second level of enum globally. Should use enumitem instead.

\renewcommand{\qedsymbol}{$ \blacksquare $}

\definecolor{correct}{RGB}{37,150,190}

\mdfsetup{%
	middlelinecolor=blue!40,
	middlelinewidth=1pt,
	backgroundcolor=blue!25!purple!10!,
	roundcorner=5pt,
	%	innerleftmargin=1cm,
	%	innerrightmargin=1cm
}

%\usepackage{fourier}
\renewcommand{\rmdefault}{bch}
%\usepackage[sfdefault]{roboto}  %% Option 'sfdefault' only if the base font of the document is to be sans serif
%\usepackage[T1]{fontenc}
%\renewcommand{\familydefault}{\sfdefault}

%\pagecolor[rgb]{0,0,0} %black    % to change page to night mode
%\color[rgb]{0.5,0.5,0.5} %grey

\begin{document}
\begin{enumerate}
	\item Suppose we deal five cards from an ordinary 52-card deck.
	\begin{enumerate}
		\item What is the conditional probability that all five cards are spades, given that at least four of them are spades?
			\begin{mdframed}
				\begin{align*}
					P\left( \text{All 5 are spades } | \text{ At least 4 spades} \right) &= \frac{P\left( \text{All 5 are spades} \right)}{P\left( \text{4 spades + 1 other card} \right) + P\left( \text{All 5 are spades} \right)}\\
					&= \frac{ \frac{\binom{13}{5}}{\binom{52}{5}} }{ \frac{\binom{13}{4}  \binom{39}{1}}{\binom{52}{5}} + \frac{ \binom{13}{5}}{\binom{52}{5}} }\\
					&= \frac{\binom{13}{5}}{\binom{13}{4}  \binom{39}{1} + \binom{13}{5}}\\
					\Aboxed{{} &= \frac{3}{68}}
				\end{align*}
			\end{mdframed}
		\item What is the conditional probability that the hand contains no pairs, given that it contains no spades?
			\begin{mdframed}
				$ P\left( \text{Containing no spades} \right) = \frac{\binom{39}{5}}{\binom{52}{5}}$. $ P\left( \text{Containing no pairs and no spades} \right) $ means that for the first card chosen, any of 39 cards can be chosen, then for the second card, any of 36 cards can be chosen (excluding the other 3 cards that can make previous card a pair), etc.\\
				i.e. $ P\left( \text{Containing no pairs and no spades} \right) = \frac{39}{52} \cdot \frac{36}{51} \cdot \frac{33}{50} \cdot \frac{30}{49} \cdot \frac{27}{48} = \frac{8019}{66640}$\\
				$ \therefore  P\left( \text{Containing no pairs \emph{given} no spades} \right) = \frac{\frac{8019}{66640}}{\frac{\binom{39}{5}}{\binom{52}{5}}} \boxed{= \tfrac{2673}{4921}}$
			\end{mdframed}
	\end{enumerate}

	\item (Q6 on p.84 From B\&H)\\
	A hat contains 100 coins, where 99 are fair but one is double-headed (always landing Heads). A coin is chosen uniformly at random. The coin is flipped 7 times, and it lands Heads all 7 times. Given this information, what is the probability that the chosen coin is double-headed?
		\begin{mdframed}
			From description: $ P\left( \text{1-head coin} \right) = \frac{99}{100}$, $ P\left( \text{2-head coin} \right) = \frac{1}{100} $, $ P\left( \text{All heads $ | $ 1-sided coin} \right) = \frac{1}{2^7}$ and \\
			$ P\left( \text{All heads $ | $ 2-head coin} \right) = 1$. Therefore, by Bayes' Theorem,
			\begin{align*}
				P\left( \text{2-head coin $ | $ All heads} \right) &= \frac{P\left( \text{All heads $ | $ 2-head coin} \right) P\left( \text{2-head coin} \right)}{P\left( \text{H $ | $ 2-head coin} \right) P\left( \text{2-head coin} \right) + P\left( \text{H $ | $ 1-head coin} \right)P\left( \text{1-head coin} \right)}\\
				&= \frac{1 \frac{1}{100}}{1 \frac{1}{100} + \frac{1}{2^7}\frac{99}{100}}\\
				&= \frac{\cancel{\frac{1}{100}}\left( 1 \right)}{\cancel{\frac{1}{100}}\left( 1 + \frac{99}{2^7} \right)}\\
				\Aboxed{{} &= \tfrac{2^7}{2^7 + 99}}
			\end{align*}
		\end{mdframed}
\end{enumerate}
\end{document}